\documentclass[12pt]{article}


\title{A proposal for a more efficient system of Unix identity
  administration using existing infrastructure}

\author{Dmitri G. Brengauz\thanks{Thanks to Jason Goode, Don Miller,
    and Ray Allarakhia for feedback and support.}\\
\small End User Services; Id Administration\\
\small for International Business Machines \\
%\small 6300 Diagonal Hwy\\
%\small Boulder, CO 80301-6108\\
\small \texttt{dmitri@us.ibm.com}\\
}

\date{07 June 2011}

\begin{document}
\maketitle

\begin{abstract}

Abstract goes here, if it really needs it.

\end{abstract}

\section{Introduction}

When I was first hired here, I looked in amazement at the list of over
200 Sun servers that  was attached to my access approval form.  The
sheer computing power that this represents is staggering.  However,
after working with them for even a short time, one quickly sees that
this  capacity seldom goes to waste:  many of these servers are
already running as hard as they can.


These are the dual challenges of a Unix account: managing a
great number of servers, and doing so in a matter that leaves the
smallest possible administrative footprint.  

Currently, the way these servers are administered is through an agent
establishing an interactive Secure Shell session to the server from
the agent's laptop, which is itself connected to the client's LAN
through a SOCS proxy.  

User management is by means of userland\footnote{`` \textbf{userland}:
  n.\\ Anywhere outside the kernel. ``That code belongs in userland.''
  This term has been in common use among Unix kernel hackers since at
  least 1985, and may have have originated in that community. The
  earliest sighting was reported from the usenet group
  net.unix-wizards. (Raymond, Eric S., \textit{et al.} \textit{The
    Jargon File} version 4.4.7, 29 Dec 2003}  tools that are a part of the standard
operating system installation.  Users are created and modified with
the \texttt{useradd} command.  The \texttt{passwd} utility is used to
manage authentication tokens.  It remains a simple script to
consistently modify the \texttt{/etc/password} file, not much changed
since AT\&T's Version 6 UNIX of May, 1975.\footnote{ (2007,May
  8). passwd(5) FreeBSD manual page.}







\section{Existing Alternatives}

Other accounts have used 
Ruby --easily accessible.  Self-documenting code.



\section{Multipass}

Wasting bandwidth by sending packets to keep streams open while the
server patiently waits for the human to type in the password.   


Cross-platform:  tested it at home on a FreeBSD workstation, an  AT\&T
UNIX descendant that has little in common with the ``Windows NT 5.1''
operating system on EUS Admin workstations.


\section{Thor , frameworks, and beyond}

Routine maintenance operations, such as locking accounts, could be
scheduled to run on off-peak times using Unix's built-in \texttt{cron}
utility, allowing agents to prioritize clients' most urgent needs. 


There are many Ruby tools in production that facilitate the concurrent
management of several servers, the most famous one probably being the
Capistrano\footnote{https://github.com/capistrano/capistrano/wiki} framework.

Virtual server: a project like this has very low initial overhead,
greater resources only become required as the tools are used by a
greater amount of users.  


Open standards make possible the taking of a system of imposing
complexity, and segmenting it into manageable chunks.



Options made possible not by an outside vendor whose production cycles
determine what is possible, but by an open framework 


\section{Conclusion}




\end{document}
