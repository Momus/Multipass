\documentclass[12pt]{article}


\title{A proposal for a more efficient system of Unix identity
  administration using existing infrastructure}

\author{Dmitri G. Brengauz\thanks{Thanks to Jason Goode, Don Miller,
    and Ray Allarakhia for feedback and support.}\\
\small End User Services; Id Administration\\
\small for International Business Machines \\
%\small 6300 Diagonal Hwy\\
%\small Boulder, CO 80301-6108\\
\small \texttt{dmitri@us.ibm.com}\\
}

\date{07 June 2011}

\begin{document}
\maketitle

\begin{abstract}

Abstract goes here, if it really needs it.

\end{abstract}

\section{Introduction}

The problem at hand:

When I was first hired here, I looked in amazement at the list of over
200 Sun servers, which was attached to my access approval form.  The
sheer computing power that this represents is staggering.  However,
for all their robustness, the clients soon seem 

the \texttt{passwd} utility, which is still a simple script to
consistently modify the \texttt{/etc/password} file, not much changed
since AT\&T's Version 6 UNIX of May, 1975.\footnote{ (2007,May
  8). passwd(5) FreeBSD manual page.}




\section{Existing Alternatives}

Other accounts have used 
Ruby --easily accessible.  Self-documenting code.



\section{Multipass}

Wasting bandwidth by sending packets to keep streams open while the
server patiently waits for the human to type in the password.   


Cross-platform:  tested it at home on a FreeBSD workstation, an  AT\&T
UNIX descendant that has little in common with the ``Windows NT 5.1''
operating system on EUS Admin workstations.


\section{Thor , frameworks, and beyond}

Routine maintenance operations, such as locking accounts, could be
scheduled to run on off-peak times using Unix's built-in \texttt{cron}
utility, allowing agents to prioritize clients' most urgent needs. 


There are many Ruby tools in production that facilitate the concurrent
management of several servers, the most famous one probably being the
Capistrano\footnote{https://github.com/capistrano/capistrano/wiki} framework.

Virtual server: a project like this has very low initial overhead,
greater resources only become required as the tools are used by a
greater amount of users.  


Open standards make possible the taking of a system of imposing
complexity, and segmenting it into manageable chunks.



Options made possible not by an outside vendor whose production cycles
determine what is possible, but by an open framework 


\section{Conclusion}




\end{document}
